% !TEX encoding = UTF-8
% !TEX TS-program = pdflatex
% !TEX root = ../tesi.tex
	
%**************************************************************
\chapter{Introduction}
\label{chap:introduction}
%**************************************************************


%**************************************************************
\section{THRON}

\begin{figure}[htbp]
\begin{center}
\includegraphics[width=7cm]{immagini/pictures/thron.png} 
\caption{THRON logo.}
\end{center}
\end{figure}

THRON S.p.A.\footnote{\url{https://www.thron.com}} is an Italian company that develops a \emph{\gls{marketing DAM}}\textsubscript{g} software. Founded in 2000 as New Vision by CEO Nicola Meneghello and CTO Dario De Agostini, it was one of the pioneers at delivering web applications in Italy. \\
In 2004 the software house launched 4ME, a cloud based service for content management which was the precursor of the current company-homonym product. \\
Nowadays THRON has four premises around the globe and innovation remains one of its core values.\\
In the following I will refer to the company as THRON and to the product as Thron to distinguish them.

%**************************************************************
\section{Birth of the project}

THRON's homonymous product is a \emph{marketing DAM}, a software application that can be seen as a centralized archive to manage multimedia content; any type of file can be stored, but contextually images are predominant. A DAM provides various features such as advanced item search, property management and tools to share content on social media and web platforms. \\
The core feature of a DAM is its ability to categorize the indexed media, making it easy for the user to find and edit objects with specific properties or tagged in a certain way. Thron makes this process even simpler thanks to its intelligence, a system that automatically elaborates the inserted media and suggests how to tag and categorize it. \\
Currently Thron's intelligence is powered by general purpose \emph{\gls{artificial intelligence}}\textsubscript{g} tools, hence the ability to recognize only a limited set of classes/features. Since most of Thron's users are big brand-name companies, teaching the system how to detect custom classes relevant for each client would improve greatly the software's relevance on the market. \\
As stated above, the vast majority of the stored media is in an image format; this both determines the problem and its solution: what is needed in Thron is an object detection system, and given the considerable database already available the most straightforward way to create it is to follow a deep learning\footnote{I will exhaustively explain what deep learning is and why it's the best solution for this problem in chapter two.} method.



%**************************************************************
\section{Content organization}

\begin{description}
    \item[{\hyperref[cap:deep-learning]{The second chapter}}] briefly covers the theory behind \emph{deep learning} and explains why it proved to be the best approach to build a custom \emph{object detection} application for Thron.
    
    \item[{\hyperref[cap:internship]{The third chapter}}] covers the goals of my project, showing its requirements and the work schedule.
    
    
    \item[{\hyperref[cap:design-development]{The fourth chapter}}] briefly showcases the technologies I used and explains the reasons why I chose them. Furthermore, the chapter it covers my main design choices and application architecture.
    
    \item[{\hyperref[cap:test-validation]{The fifth chapter}}] explains my choices on the matter of tests and validation.
    
    \item[{\hyperref[cap:conclusion]{The sixth chapter}}] compares my initial goals to what was actually achievable and explains the problems that occurred during development.
\end{description}

This works adheres to the following typographic conventions:
\begin{itemize}
	\item acronyms, abbreviations, ambiguous and field-specific terms are defined in the glossary at the end of the document;
	\item the first occurrence of glossary terms is emphasized as follows : \emph{glossary term}\textsubscript{g};
	\item technical terms are emphasized as follows: \emph{technical term}.
\end{itemize}