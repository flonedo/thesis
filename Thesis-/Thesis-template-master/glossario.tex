\selectlanguage{english}
%**************************************************************
% Acronimi
%**************************************************************
%\renewcommand{\acronymname}{Acronimi e abbreviazioni}

\renewcommand{\acronymname}{Acronyms}


    

%**************************************************************
% Glossario
%**************************************************************
%\renewcommand{\glossaryname}{Glossario}

\newglossaryentry{marketing DAM}
{
	name=\glslink{marketing DAM}{Marketing DAM},
	text=marketing DAM,
	sort=marketing dam,
	description={Acronym of Digital Asset Management; a DAM is a software application to store, categorize and access multimedia content.}
}

\newglossaryentry{artificial intelligence}
{
	name=\glslink{artificial intelligence}{Artificial Intelligence},
	text=artificial intelligence,
	sort=artificial intelligence,
	description={Umbrella term for theory and development of computer systems able to perform tasks normally requiring human intelligence, such as visual perception, speech recognition, decision-making, and translation between languages. Machine learning and deep learning are sub-fields of artificial intelligence.
\begin{figure}[htbp]
\begin{center}
\includegraphics[width=7cm]{immagini/pictures/ai-ml-dl.png} 
\caption{Relationship between artificial intelligence, machine learning and deep learning.}
\end{center}
\end{figure}}
}

\newglossaryentry{deep learning}
{
	name=\glslink{deep learning}{Deep Learning},
	text=deep learning,
	sort=deep learning,
	description={Part of a broader family of machine learning methods based on learning data representation, as opposed to task-specific algorithms. Deep learning models are inspired by the way biological brains compute data, and as such learn features by repeatedly experiencing them in a training process.}
}

\newglossaryentry{TensorFlow}
{
	name=\glslink{TensorFlow}{Tensorflow},
	text=TensorFlow,
	sort=tensorflow,
	description={Open source framework mainly used for machine learning applications developed by Google. TensorFlow website: \url{https://www.tensorflow.org/}.
\begin{figure}[htbp]
\begin{center}
\includegraphics[width=7cm]{immagini/pictures/tf.png} 
\caption{TensorFlow logo.}
\end{center}
\end{figure}}
}

\newglossaryentry{MXNet}
{
	name=\glslink{MXNet}{MXNET},
	text=MXNet,
	sort=mxnet,
	description={Open source framework for deep learning applications developed by Apache. MXNet website: \url{https://mxnet.apache.org/}.
\begin{figure}[htbp]
\begin{center}
\includegraphics[width=7cm]{immagini/pictures/mxn.png} 
\caption{MXNet logo.}
\end{center}
\end{figure}}
}

\newglossaryentry{convolutional neural network}
{
	name=\glslink{convolutional neural network}{Convolutional Neural Network},
	text=convolutional neural network,
	sort=convolutional neural network,
	description={In deep learning, class of deep, feed-forward artificial neural networks, most commonly applied to analyzing visual imagery. Deep neural networks for object detection are SSD, YOLO and R-CNN.}
}

\newglossaryentry{YOLO v3}
{
	name=\glslink{YOLO v3}{YOLO v3},
	text=YOLO v3,
	sort=yolo v3,
	description={Acronym of You Only Look Once; YOLO v3 is a convolutional neural network developed by Joseph Redmon. YOLO website: \url{https://pjreddie.com/darknet/yolo/}}
}

\newglossaryentry{machine learning}
{
	name=\glslink{machine learning}{Machine Learning},
	text=machine learning,
	sort=machine learning,
	description={Field of artificial intelligence that uses statistical techniques to give computer systems the ability to "learn" (e.g., progressively improve performance on a specific task) from data, without being explicitly programmed. Differently from deep learning methods, machine learning applications still need the problem's boundaries to be defined in the code by a human.}
}

\newglossaryentry{computer vision}
{
	name=\glslink{computer vision}{Computer Vision},
	text=computer vision,
	sort=computer vision,
	description={Interdisciplinary field that deals with how computers can be made to gain high-level understanding from digital images or videos. Computer vision technology is used to automate tasks that the human vision can do, such as instance counting, object and face recognition and video surveillance.}
}

\newglossaryentry{dataset}
{
	name=\glslink{dataset}{Dataset},
	text=dataset,
	sort=dataset,
	description={Collection of data, commonly with statistic properties. Popular open source datasets used in computer vision are MNIST, COCO, ImageNet and VOC.}
}

\newglossaryentry{epoch}
{
	name=\glslink{epoch}{Epoch},
	text=epoch,
	sort=epoch,
	description={Epoch is when the entire dataset goes through the network forth (and back) once. When training you a network you commonly perform hundreds of epochs.}
}

\newglossaryentry{overfitting}
{
	name=\glslink{overfitting}{Overfitting},
	text=overfitting,
	sort=overfitting,
	description={Overfitting is a problem occurring in training a network when, due too long training or poor data samples, the models learns patterns that aren't actually relevant and therefore fails during inference. An overfitted model would predict exactly the content of sample data and fail on everything else because it adjusted too tightly to the examples.}
}

\newglossaryentry{mean average precision}
{
	name=\glslink{mean average precision}{Mean Average Precision},
	text=mean average precision,
	sort=mean average precision,
	description={Accuracy evaluation metric for object detection models, also known with the acronym mAP. To calculate how well your network performs, mAP considers both the number of positives and actually how well these overlap with the ground truth bounding box. This second task is calculated using an IoU threshold.}
}

\newglossaryentry{intersection over union}
{
	name=\glslink{intersection over union}{Intersection Over Union},
	text=intersection over union,
	sort=intersection over union,
	description={Also known as IoU, measure to determine how well a detected bounding box overlaps with the ground truth.}
}

\newglossaryentry{loss function}
{
	name=\glslink{loss function}{Loss function},
	text=loss function,
	sort=loss function,
	description={Function to calculate the distance of the predicted value from the ground truth label. Commonly used loss functions are \emph{Quadratic Loss Function} and \emph{Cross Entropy Loss Function}.}
}

\newglossaryentry{gradient descent function}
{
	name=\glslink{gradient descent function}{Gradient descent function},
	text=gradient descent function,
	sort=gradient descent function,
	description={Function based on differentiation that calculates how the weights should be changed to reduce the loss function.}
}

\newglossaryentry{back-propagation}
{
	name=\glslink{back-propagation}{Back-propagation},
	text=back-propagation,
	sort=back-propagation,
	description={Propagation of the calculated values to improve the network performance; these values are used to update the weights used by the neurons functions.}
}

\newglossaryentry{AWS}
{
	name=\glslink{AWS}{Amazon Web Services},
	text=AWS,
	sort=aws,
	description={Acronym of Amazon Web Services. A set of various services including, among others, cloud storage, computational power and artificial intelligence specific services.}
}

\newglossaryentry{Gluon CV}
{
	name=\glslink{Gluon CV}{Gluon CV},
	text=Gluon CV,
	sort=gluon cv,
	description={High level toolkit for computer vision application on MXNet Python framework. Gluon CV website: \url{https://gluon-cv.mxnet.io/}}
}

\newglossaryentry{CUDA}
{
	name=\glslink{CUDA}{CUDA},
	text=CUDA,
	sort=cuda,
	description={Compute Unified Device Architecture. Parallel computing platform and programming model invented by NVIDIA; it is vastly used in deep learning frameworks and applications.}
}

\newglossaryentry{Pip}
{
	name=\glslink{Pip}{Pip},
	text=Pip,
	sort=pip,
	description={Pip Installs Packages. Package management system used to install and manage packages written in Python. It is mostly used for packages on PypI.}
}

\newglossaryentry{PyPI}
{
	name=\glslink{PyPI}{Pypi},
	text=PyPi,
	sort=pypi,
	description={Python Package Index. Official third party repository for Python packages. PyPI website: \url{https://pypi.org/}}
}

\newglossaryentry{Jupyter notebook}
{
	name=\glslink{Jupyter notebook}{Jupyter notebook},
	text=Jupyter notebook,
	sort=jupyter notebook,
	description={Web-based interactive computational environment for creating Jupyter notebooks documents.}
}

\newglossaryentry{Docker}
{
	name=\glslink{Docker}{Docker},
	text=Focker,
	sort=docker,
	description={Computer program that performs operating-system-level virtualization, also known as "containerization". A Docker container can be compared to a virtual machine, but much lighter. Docker website: \url{https://www.docker.com/}}
}

\newglossaryentry{YAML}
{
	name=\glslink{YAML}{YAML},
	text=YAML,
	sort=yaml,
	description={YAML Ain't Markup Language. Human-readable serialization language, commonly used in configuration files.}
}

\newglossaryentry{OpenAPI}
{
	name=\glslink{OpenAPI}{OpenAPI},
	text=OpenAPI,
	sort=openapi,
	description={Open Application Programming Interface. Publicly available interface to interact with web services; it often serves as documentation and guideline for service implementation (API first).}
}

\newglossaryentry{Swagger}
{
	name=\glslink{Swagger}{Swagger},
	text=Swagger,
	sort=swagger,
	description={Online tool to declare OpenAPIs. Swagger website: \url{https://swagger.io/}}
}

\newglossaryentry{IntelliJ IDEA}
{
	name=\glslink{IntelliJ IDEA}{IntelliJ IDEA},
	text=IntelliJ IDEA,
	sort=intellij idea,
	description={Java Integrated development Environment developed by JetBrains. It provides plugins for several languages such as Python and Scala. IntelliJ IDEA website: \url{https://www.jetbrains.com/idea/}}
}

\newglossaryentry{Postman}
{
	name=\glslink{Postman}{Postman},
	text=Postman,
	sort=postman,
	description={API development tool; it features functionality to send HTTP request and edit their content. Postman website: \url{https://www.getpostman.com/}}
}

\newglossaryentry{Robo 3T}
{
	name=\glslink{Robo 3T}{Robo 3T},
	text=Robo 3T,
	sort=robo 3t,
	description={MongoDB client. It allows to do queries and manage the database content. Robo 3T website: \url{https://robomongo.org/}}
}

\newglossaryentry{MongoDB}
{
	name=\glslink{MongoDB}{MongoDB},
	text=MongoDB,
	sort=mongodb,
	description={Open source, non-relational, document-oriented database program. MongoDB website: \url{https://www.mongodb.com/}}
}

\newglossaryentry{GitLab}
{
	name=\glslink{GitLab}{GitLab},
	text=GitLab,
	sort=gitlab,
	description={Web based Git repository. GitLab website: \url{https://about.gitlab.com/}}
}

\newglossaryentry{SageMaker}
{
	name=\glslink{SageMaker}{SageMaker},
	text=SageMaker,
	sort=sagemaker,
	description={Web platform provided by Amazon to build, train and deploy machine learning models. It both supports popular frameworks and custom docker containers. Amazon SageMaker website: \url{https://aws.amazon.com/sagemaker/}}
}

\newglossaryentry{Flask}
{
	name=\glslink{Flask}{Flask},
	text=Flask,
	sort=flask,
	description={Python server framework. Flask website: \url{http://flask.pocoo.org/}}
}

\newglossaryentry{JSON lines file}
{
	name=\glslink{JSON lines file}{JSON lines file},
	text=JSON lines,
	sort=json lines,
	description={A Json lines file is a *.json file that contains a valid Json on each line.}
}

\newglossaryentry{RESTful API}
{
	name=\glslink{RESTful API}{RESTful API},
	text=RESTful API,
	sort=restful api,
	description={Representational State Transfer Application Program Interface. Web services that uses HTTP requests to perform transactions.}
}

\newglossaryentry{polling}
{
	name=\glslink{polling}{Polling},
	text=polling,
	sort=polling,
	description={In an asynchronous context, process where a client actively checks whether the requested resource is available by sending requests (e.g. client sending a HTTP Get requests to the server).}
}

\newglossaryentry{callback}
{
	name=\glslink{callback}{Callback},
	text=callback,
	sort=callback,
	description={In an asynchronous context, process where the server notifies the client when the requested resource is ready.}
}
