
%**************************************************************
% Acronimi
%**************************************************************
\renewcommand{\acronymname}{Acronimi e abbreviazioni}

\newacronym[description={\glslink{apig}{Application Program Interface}}]
    {api}{API}{Application Program Interface}

\newacronym[description={\glslink{umlg}{Unified Modeling Language}}]
    {uml}{UML}{Unified Modeling Language}

\newacronym[description={\glslink{Artificial intelligence}{Artificial intelligence}}]
	{AI}{Artificial Intelligence}



%**************************************************************
% Glossario
%**************************************************************
%\renewcommand{\glossaryname}{Glossario}

\newglossaryentry{apig}
{
    name=\glslink{api}{API},
    text=Application Program Interface,
    sort=api,
    description={in informatica con il termine \emph{Application Programming Interface API} (ing. interfaccia di programmazione di un'applicazione) si indica ogni insieme di procedure disponibili al programmatore, di solito raggruppate a formare un set di strumenti specifici per l'espletamento di un determinato compito all'interno di un certo programma. La finalità è ottenere un'astrazione, di solito tra l'hardware e il programmatore o tra software a basso e quello ad alto livello semplificando così il lavoro di programmazione}
}

\newglossaryentry{umlg}
{
    name=\glslink{uml}{UML},
    text=UML,
    sort=uml,
    description={in ingegneria del software \emph{UML, Unified Modeling Language} (ing. linguaggio di modellazione unificato) è un linguaggio di modellazione e specifica basato sul paradigma object-oriented. L'\emph{UML} svolge un'importantissima funzione di ``lingua franca'' nella comunità della progettazione e programmazione a oggetti. Gran parte della letteratura di settore usa tale linguaggio per descrivere soluzioni analitiche e progettuali in modo sintetico e comprensibile a un vasto pubblico}
}

\newglossaryentry{marketing DAM}
{
	name=\glslink{marketing DAM}{Marketing DAM},
	text=marketing DAM,
	sort=marketing dam,
	description={Acronym of Digital Asset Management, a DAM is a software application to store, categorize and access multimedia content.}
}

\newglossaryentry{artificial intelligence}
{
	name=\glslink{artificial intelligence}{Artificial Intelligence}{AI},
	text=artificial intelligence,
	sort=artificial intelligence,
	description={Umbrella term for theory and development of computer systems able to perform tasks normally requiring human intelligence, such as visual perception, speech recognition, decision-making, and translation between languages.}
}

\newglossaryentry{deep learning}
{
	name=\glslink{Deep Learning}{deep learning},
	text=deep learning,
	sort=deep learning,
	description={Part of a broader family of machine learning methods based on learning data representation, as opposed to task-specific algorithms. Deep learning models are inspired by the way biological brains compute data, and as such learn features by repeatedly experiencing them in a training process.}
}

\newglossaryentry{Tensorflow}
{
	name=\glslink{TensorFlow},
	text=tensorflow,
	sort=tensorflow,
	description={Open source framework mainly used for machine learning applications developed by Google.}
}

\newglossaryentry{MXNet}
{
	name=\glslink{MXNet},
	text=mxnet,
	sort=mxnet,
	description={Open source framework for deep learning applications developed by Apache.}
}

\newglossaryentry{convolutional neural network}
{
	name=\glslink{convolutional neural network}{CNN}{Convolutional Neural Network},
	text=convolutional neural network,
	sort=convolutional neural network,
	description={In deep learning, class of deep, feed-forward artificial neural networks, most commonly applied to analyzing visual imagery.}
}

\newglossaryentry{YOLO v3}
{
	name=\glslink{YOLO v3}{YOLO},
	text=yolo v3,
	sort=yolo v3,
	description={Acronym of You Only Look Once, YOLO v3 is a convolutional neural network developed by Joseph Redmon.}
}

\newglossaryentry{machine learning}
{
	name=\glslink{machine learning}{Machine Learning}{ML},
	text=machine learning,
	sort=machine learning,
	description={Field of artificial intelligence that uses statistical techniques to give computer systems the ability to "learn" (e.g., progressively improve performance on a specific task) from data, without being explicitly programmed.}
}

\newglossaryentry{computer vision}
{
	name=\glslink{computer vision}{Computer Vision}{CV},
	text=computer vision,
	sort=computer vision,
	description={Interdisciplinary field that deals with how computers can be made to gain high-level understanding from digital images or videos; computer vision technology is used to automate tasks that the human vision can do.}
}

\newglossaryentry{dataset}
{
	name=\glslink{dataset},
	text=dataset,
	sort=dataset,
	description={Collection of data, commonly with statistic properties. Popular open source datasets used in computer vision are MNIST, COCO, ImageNet and Pascal VOC.}
}

\newglossaryentry{epoch}
{
	name=\glslink{epoch}{epochs},
	text=epoch,
	sort=epoch,
	description={Epoch is when the entire dataset goes through the network forth (and back) once. When training you a network you commonly perform hundreds of epochs.}
}

\newglossaryentry{overfitting}
{
	name=\glslink{overfitting},
	text=overfitting,
	sort=overfitting,
	description={Overfitting is a problem occurring in training a network when, due too long training or poor data samples, the models learns patterns that aren't actually relevant and therefore fails during inference. An overfitted model would predict exactly the content of sample data and fail on everything else because it adjusted too tightly to the examples.}
}

\newglossaryentry{mean average precision}
{
	name=\glslink{mean average precision}{mAP},
	text=mean average precision,
	sort=mean average precision,
	description={Accuracy evaluation metric for object detection models. To calculate how well your network performs it considers both the number of positives and actually how well these overlap with the ground truth bounding box. This second task is calculated using an IoU threshold.}
}

\newglossaryentry{intersection over union}
{
	name=\glslink{intersection over union}{IoU},
	text=intersection over union,
	sort=intersection over union,
	description={Measure to determine how well a detected bounding box overlaps with the ground truth.}
}

\newglossaryentry{loss function}
{
	name=\glslink{loss function},
	text=loss function,
	sort=loss function,
	description={Function to calculate the distance of the predicted value from the ground truth label. Commonly used loss functions are \emph{Quadratic Loss Function} and \emph{Cross Entropy Loss Function}.}
}

\newglossaryentry{gradient descent function}
{
	name=\glslink{gradient descent function},
	text=gradient descent function,
	sort=gradient descent function,
	description={Function based on differentiation that calculates how the weights should be changed to reduce the loss function.}
}

\newglossaryentry{back-propagation}
{
	name=\glslink{back-propagation},
	text=back-propagation,
	sort=back-propagation,
	description={Propagation of the calculated values to improve the network performance; these values are used to update the weights used by the neurons functions.}
}

\newglossaryentry{AWS}
{
	name=\glslink{AWS},
	text=AWS,
	sort=aws,
	description={Amazon Web Services. A set of various services including, among others, cloud storage, computational power and artificial intelligence specific services.}
}


